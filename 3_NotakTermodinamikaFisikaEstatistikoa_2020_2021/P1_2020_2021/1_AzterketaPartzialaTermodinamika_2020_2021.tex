\documentclass[10pt]{article}              % Book class in 11 points
\parindent0pt  \parskip10pt             % make block paragraphs
\usepackage[dvips]{color}

\usepackage{tcolorbox}
\tcbuselibrary{breakable}

\usepackage{afterpage}

\usepackage{xcolor}
\usepackage[basque]{babel}
\selectlanguage{basque}
\usepackage[hmargin=2.5cm,vmargin=2.5cm,headheight=15pt]{geometry}
\usepackage{graphicx}
\usepackage{amsmath,amsfonts,amssymb}
%\usepackage{picins}

\begin{document}                        % End of preamble, start of text.




\section*{2020-2021 Ikasturtea\\
\textit{Termodinamika eta Fisika Estatistikoa irakasgaia}\\
1. azterketa partziala, Termodinamika\\
(2020ko urtarrilaren 22a)}

\vspace{1.25cm}


\begin{enumerate}



\item 
	\begin{enumerate}
	\item Frogatu bero-iturri batekin ukipen termikoan gertatzen den bi bolumenen arteko gas baten zabaltzean, bero gehiago xurgatzen duela gasak zabaltzea itzulgarria bada itzulezina bada baino.
%	\item[] 	
	\item Frogatu $V$ eta $T$ konstantepeko prozesu itzulezinaren ondorioz edozein sistemaren energia askeak ($F$) behera egiten duela.
%	\item[] 
	\item Uraren, ur likidoaren, $\alpha$ zabalkuntza-koefizientea negatiboa da $0^{\circ} \textrm{C}< T < 4^{\circ} \textrm{C}$ tenperatura-tartean.\\ 
	Frogatu ezen urak beroa xurgatzen duela era isotermo itzulgarrian konprimituz gero $3^{\circ}$ Can.
%	\item[] 
%	\item Likido bati dagokion lurrun-presioaren adierazpena honako hau da: $\ln p = A - \frac{B}{T} + C\ln T$. Adierazpen horretan $(A, B, C)$ konstanteak dira. Lortu $\Delta H_{l\to g}$.
%	\item[] 
	\item Sistema bati dagokion askatasun-gradu batekin lotutako aldagai estentsibo eta intebtsiboa $X$ eta $Y$ dira, hurrenez hurren.\\
	Askatasun-gradu horrekin lotutako egoera-ekuazioa honako hau da: $ X = \frac{c}{T} Y $. Adierazpen horretan $c$ da konstante ezaguna.\\
	Lortu askatasun-graduarekin lotutako bero-ahalmenen arteko lotura (erlazioa).
	\end{enumerate}
	
\vspace{1.25cm}

\item 
\begin{enumerate}
\item Lortu gas ideal monoatomiko baten $F\equiv U\left[T\right]=F\left(T,V,N\right)$ potentzial termodinamikoa. Honako hau:  

$$F = N R T \left\{ \frac {F_{0}}{N_{0}RT_{0}} - \ln \left[ \left( \frac {T}{T_{0}} \right) ^ {3/2} \left( \frac {V}{V_{0}} \right) \left(\frac{N}{N_{0}} \right)^{-1} \right] \right\}$$

\item[]

\item Gas ideal monoatomikoren horren 2 mol $(p_{i}, V_{i})$ hasierako oreka-egoeratik $(p_{f}=B^{2}p_{i}, V_{f}=\frac{V_{i}}{B})$ ($B$ konstantea da) amaierako oreka-egoerara eraman ditugu. Foko termikoa ($T_{C}$ tenperaturako bero-iturria) eta lan-fokoa erabilgarriak dira. Lortu lan-fokoari eman diezaiokegun lan maximoa. $B$, $p_{i}$ eta $T_{C}$ parametroen balioak finkaturik daudela, $V_{i}$ bolumenaren zein baliok egingo du lana positibo?
\end{enumerate}

\vspace{1.25cm}

\item Gomazko banda baten egoera-ekuazioa honako hau da:

$$ \tau = a\, T \left[ \left(\frac{L}{L_{0}}\right) - \left(\frac{L_{0}}{L}\right)^{2} \right]$$
Adierazpen horretan, $\tau$ da tentsioa, eta $L$, luzera. $C_{L}$, $a$ eta $L_{0}$ konstante ezagunak dira.
\begin{enumerate}
\item Froga ezazu tenperaturaren funtzioa baino ez dela barne-energia.
%\item[]
\item Banda luzatzen da, era isotermo itzulgarrian, $L = L_{0}$-tik $L = 2L_{0}$-ra.\\
Prozesuan tenperatura 300 K da.\\
Lortu bandaren gainean egindako lana $(W)$ eta trukatu behar izan den beroa $(Q)$.
%\item[]
\item Banda hori isoentropikoki luzatu izan balitz, zer tenperatura litzateke bukaerakoa?
\end{enumerate}

\vspace{5cm}

%\item Aztertu beharreko sistema honako hau da: 1 atm-ean dagoen gordailuan sartu dugun substantzia baten lurruna. Gordailua 400 K-ean dagoen bero-iturriarekin ukipenean jarri dugu eta, tenperatura konstate mantenduz, 10 atm-raino konprimitu da.\\
%    
%    Ezaguna da substantzia hori 300 K-ean eta 1 atm-eko presioan lurrunduko dela.\\
%    
%    Lurruntze-prozesuari dagokion entropia-aldaketa da malda negatiboko lerro zuzena, hain zuzen, honako hau: $\Delta s=-0.0676$ (cal/K$^2$ mol) $\times T + 37.856$ (cal/mol K).\\
%    
%    Substantziaren likidoaren bolumen espezifikoak ondoko egoera-ekuazioari segitu dio: $v=v_{0}(1+aT)$; $a=10^{-6}$ K$^{-1}$.
%    
%\begin{enumerate} 
%\item Irudikatu prozesua $p/T$ diagraman, ezagunak diren puntu guztiak kokatuz.
%\item[]
%\item Kalkulatu sistemaren entropia-aldaketa.
%\item[]
%\item Kalkulatu fase-trantsizioan gertatu den barne-energiaren aldaketa.
%\item[]
%\item Puntu hirukoitzaren tenperatura 200 K bada, nola kalkulatuko zenuke puntu hirukoitzaren presioa? Azaldu.
%\end{enumerate}



\item 10 eta 15 l-ko gordailuak finkoa, adiabatikoa eta iragaztezina den hormak banandu ditu. \\Lehenengo gunean SO$_{2}$ gasa sartu dugu: 288 K-ean eta 2 atm-n. \\Bigarrenean, NO gasa; kasu honetan, 300 K-ean eta 1 atm-ean\\
    
    
\begin{enumerate} 
\item Diatermano eta higikor bihurtu dugu bereizte-horma.\\
Lortu unibertsoari dagokion entropia-aldaketa.
\item[]
\item[]
\item Hasierako baldintzetatik abiatuz, bereizte-horma bat-batean kendu dugu.\\
Lortu sistema unibertsoari dagokion entropia-aldaketa.
\item [] EZ DUZU KALKULURIK EGIN BEHAR.\\
AZALDU ZEIN DEN BUKAERAKO EGOERA ETA ESAN ZEIN DIREN UNIBERTSOAREN ENTROPIA-ALDAKETAREN ATALAK.\\
Nahi baduzu, idatzi entropia-aldaketaren adierazpena.\\


Kasu honetan, \textbf{nahastura-entropiaren} definizioa erabili beharko duzu, honako hau:\\

\begin{flushright}
\begin{minipage}[c]{0.75\textwidth}
\small{It should, perhaps, be noted that equation 3.37 is integrable term by term, despite our injunction (in Example 3) that such an approach generally is not possible. The segregation of the independent variables $u$ and $v$ in separate terms in equation 3.37 is a fortunate but unusual simplification which permits term by term integration in this special case. A mixture of two or more simple ideal gases $-$ a "multicomponent simple ideal gas" $-$ is characterized by a fundamental equation which is most simply written in parametric form, with the temperature $T$ playing the role of the parametric variable.

$$
\begin{array}{l}
S=\sum_{J} N_{j} s_{j 0}+\left(\sum_{j} N_{j} c_{j}\right) R \ln \frac{T}{T_{0}}+\sum_{j} N_{j} R \ln \left(\frac{V}{N_{j} v_{0}}\right) \\
U=\left(\sum_{j} N_{j} c_{j}\right) R T
\end{array}
$$

Elimination of $T$ between these equations gives a single equation of the standard form $S=S\left(U, V, N_{1}, N_{2}, \ldots\right) .$

Comparison of the individual terms of equations 3.39 with the expression for the entropy of a single-component ideal gas leads to the following interpretation (often referred to as Gibbs's Theorem). The entropy of $a$ mixture of ideal gases is the sum of the entropies that each gas would have if it alone were to occupy the volume $V$ at temperature $T .$ The theorem is, in fact, true for all ideal gases (Chapter 13 ).

It is also of interest to note that the first of equations 3.39 can be written in the form
$$
S=\sum_{j} N_{j} s_{j 0}+\left(\sum_{j} N_{j} c_{j}\right) R \ln \frac{T}{T_{0}}+N R \ln \frac{V}{N v_{0}} \boxed{- R \sum_{j} N_{j} \ln \frac{N_{j}}{N}}
$$


and the last term is known as the "entropy of mixing." \textit{It represents the difference in entropies between that of a mixture of gases and that of $a$ collection of separate gases each at the same temperature and the same density as the original mixture} $N_{j} / V_{j}=N / V,$ (and hence at the same pressure as the original mixture); see Problem $3.4-15 .$ The close similarity, and the important distinction, between Gibbs's theorem and the interpretation of the entropy of mixing of ideal gases should be noted carefully by the reader. An application of the entropy of mixing to the problem of isotope separation will be given in Section 4.4 (Example 4 ).}

\end{minipage}
\\

\end{flushright}
%\item[]
\item[BONUS] Gordailuetako gasak berdinak izanik, nola aldatuko lirateke aurreko emaitzak?
\end{enumerate}

\item[]
\item[]
\item[] (b) atalek ariketaren \%25eko balioa du.\\
  BONUS atalak ez du kontatzen.

\end{enumerate}
%


%}
\end{document}
